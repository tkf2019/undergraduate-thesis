% !TeX root = ../thesis.tex

\chapter{软件实现}

\section{QEMU}

QEMU \cite{qemu} 为操作系统和用户态程序提供虚拟的执行环境,通过动态的二进制转换,模拟 CPU 的行为,同时支持多种外设的仿真,在系统开发中扮演着重要角色。
QEMU 支持模拟 RISC-V 运行环境,通过对 QEMU 的修改和测试,我们可以不断完善设计草案。对 QEMU 的修改主要分为四个方面:

\begin{itemize}
    \item 指令翻译:引入对 UIPI 指令的译码和执行;
    \item CPU 状态:维护 CSR 寄存器等 CPU 状态;
    \item 内存读写:UIPI 指令需要直接访问物理内存和 UINTC 外设,直接调用 void cpu\_physical\_memory\_rw(hwaddr addr, void *buf, hwaddr len, bool is\_write) 函数完成对物理地址的读写;
    \item 核间中断:实现 UINTC 并向各个核发送中断。
\end{itemize}

\subsection{指令翻译}

QEMU 翻译一条指令的大致过程为:从客户机指令(Guest Instructions)到中间码(TCG,Tiny Code Generator),最后再到宿主机指令(Host Instructions)。
QEMU 的翻译机制类似于 CPU 流水线中的译码阶段,需要定义模式串来帮助 QEMU 在执行到某一指令时调用对应的辅助函数。模式串的定义位于 target/riscv/insn32.decode :

\lstset{language=C}
\begin{lstlisting}
uipi_send       0000000  00000 ..... 010 ..... 1111011 @r2
uipi_read       0000001  00000 ..... 010 ..... 1111011 @r2
uipi_write      0000010  00000 ..... 010 ..... 1111011 @r2
uipi_activate   0000011  00000 ..... 010 ..... 1111011 @r2
uipi_deactivate 0000100  00000 ..... 010 ..... 1111011 @r2
\end{lstlisting}

以 uret 这条指令为例,在 target/riscv/insn\_trans 目录下,有各种指令的翻译过程,主要用来将指令解析的结果(寄存器,立即数等)传递给辅助函数,将客户机指令拆解为宿主机指令来模拟目标指令的功能。
对于 uret 指令的执行涉及到较多 CPU 状态的变化,会对 pc ,CSR 等产生影响, 辅助函数的定义位于 target/riscv/helper\.h,通过宏定义 DEF\_HELPER\_x 来声明辅助函数,例如:

\lstset{language=C}
\begin{lstlisting}
DEF_HELPER_1(uret, tl, env)
DEF_HELPER_4(csrrw, tl, env, int, tl, tl)
\end{lstlisting}

其中第一个参数对应辅助函数的名称 ,第二个参数代表函数的返回值类型(tl 表示 target\_ulong),后面的参数都是辅助函数传入的参数类型。有了以上的参考,我们可以定义其他辅助函数:

\lstset{language=C}
\begin{lstlisting}
DEF_HELPER_2(uipi_write, void, env, tl)
void helper_uipi_write(CPURISCVState *env, target_ulong src) {
    if (uipi_enabled(env, env->suirs)) {
        uint64_t addr = UINTC_REG_HIGH(env->suicfg, SUIRS_INDEX(env->suirs));
        cpu_physical_memory_write(addr, &src, 8);
    }
}
\end{lstlisting}

\subsection{CPU 状态}

struct CPUArchState:CPU 状态结构体位于 target/riscv/cpu.h。这个结构同时考虑了 RV32、RV64、RV128 的情况,这些寄存器都是 CPU 运行时必要的状态。包括但不限于:

\begin{itemize}
    \item pc
    \item 整数、浮点寄存器堆
    \item CSR,有些寄存器是 M 态和 S 态复用的,例如 mstatus、 mip 等
    \item PMP 寄存器堆
    \item 通过 kernel\_addr、fdt\_addr 等从指定位置加载镜像
\end{itemize}

在target/riscv/cpu.h 文件末尾的表中注册 CSR 的操作函数。

中断异常、CSR 等宏定义位于 target/riscv/cpu\_bits.h ,我们需要在其中添加和 U 态有关的中断控制位。
CPU 中断异常处理函数位于 target/riscv/cpu\_helper.c 的最后,
这个函数对中断异常原因进行判断,并根据 CPU 当前的特权级做不同的处理。
这个函数只给出了 M 态和 S 态的中断异常处理,我们需要额外在此处加入委托给 U 态的中断异常处理,也就是读写 ustatus,ucause,uepc 等寄存器。

\subsection{核间中断}



\section{Linux}

\section{libc}

\section{APP}