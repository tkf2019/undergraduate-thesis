% !TeX root = ../thesis.tex

\chapter{引言}

传统的 IPC (Inter-Process Communication,进程间通信)机制包括信号、管道、命名管道 、消息队列和共享内存等\cite{modernos},其性能问题主要体现在以下几个方面:

\begin{itemize}
    \item[1.] 上下文切换开销:进程间通过内核进行通信,保存上下文和切换页表都会带来较大开销,为了解决熔断漏洞时引入的 KPTI 机制进一步增加了陷入内核的开销\cite{kpti};
    \item[2.] 数据拷贝开销:将数据块从一个进程复制到另一个进程的地址空间中,引入较高的延迟和开销;
    \item[3.] 同步互斥开销:使用锁和信号量等同步机制来保证进程之间访问共享资源的顺序,需要使用原子指令、内存屏障等硬件机制,会引入额外的开销;
    \item[4.] 安全性问题:进程和进程之间、内核和进程之间资源的共享都有可能产生数据窃取和损坏等问题。
\end{itemize}

在微内核架构中,内核值提供最基本的服务,例如虚拟存储、IPC 等,大部分的系统服务如网络协议栈、文件系统、设备驱动等都以进程的形式运行在用户空间\cite{microkernel},IPC 因此成为了微内核中最重要的部分之一,同时也是性能瓶颈之一。

用户态中断(User-Interrupt)是由用户接收和响应的中断,相比于传统的 IPC 机制,用户态直接对中断进行处理可以减少陷入内核带来的开销。无论是在宏内核还是微内核中都具有良好的应用场景。用户态中断的发送方可以是下列中的任何一种:

\begin{itemize}
    \item 外部设备:用户态驱动\cite{userdriver}相比于内核的干预更加灵活和轻便,具有良好的可移植性和安全性。引入用户态中断后,用户态驱动可以直接接收并处理外部设备发来的中断,进一步提高性能。
    \item 内核:io\_uring \cite{iouring}是 Linux 5.1 版本引入的一种异步 I/O 机制,支持请求的批量提交、数据零拷贝,已被广泛应用于各种场景。引入用户态中断后,内核在 I/O 操作完成后可以跨核向进程发送用户态中断通知,进一步提高异步唤醒的效率。
    \item 进程:seL4 的 Notification \cite{sel4}是一种轻量级的 IPC 机制,基于事件队列和快速路径,进程之间可以实现高效的异步通信,但目标进程需要通过轮询或等待的方式来处理信息。引入用户态中断后,目标进程可以在执行其他任务的同时立即接收并处理信息。
\end{itemize}

在前人工作的基础上,笔者设计并验证了 RISC-V 用户态中断扩展,通过 QEMU 模拟器对设计方案进行完善,并在 Linux 中加入了对 RISC-V 用户态中断扩展的支持;此外,笔者还在 Rocket Chip 中加入了对用户态扩展的支持,通过软件仿真验证实现,并在 zcu102 开发板上进行性能测试,证明了用户态中断无论是在模拟环境下还是在真实的硬件环境下,相比于传统的 IPC 机制可以取得明显的性能提升。

本文主要分为以下几个部分:

\begin{itemize}
    \item \textbf{背景介绍}:简要介绍 RISC-V N 扩展和 x86 用户态中断指令规范;
    \item \textbf{硬件设计方案}: RISC-V 用户态中断扩展硬件设计方案;
    \item \textbf{功能级硬件仿真模拟}:介绍在 QEMU 上关于 RISC-V 用户态中断扩展的实现细节;
    \item \textbf{硬件实现}:介绍在 Rocket Chip 上关于 RISC-V 用户态中断扩展的实现细节;
    \item \textbf{软件实现}:介绍在 Linux 和用户库上关于 RISC-V 用户态中断扩展的实现细节;
    \item \textbf{系统评估}:分别对软件实现和硬件实现进行性能测试和分析。
\end{itemize}