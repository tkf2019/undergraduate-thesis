% !TeX root = ../thesis.tex

\chapter{软件实现}

\section{Linux}

Linux \cite{linux} 是世界上最著名的开源操作系统。Linux 是标准的宏内核,可以在其上运行多种应用。目前 Linux 已经支持在 RISC-V 硬件平台上运行,也能在 QEMU 模拟的 RISC-V 硬件环境上运行。为了支持用户态中断并对其性能进行更深入的分析,我们对 Linux 6.0 版本进行了修改。针对 Linux 的修改主要分为三个部分:

\begin{itemize}
    \item 添加 UINTC 驱动并让 Linux 识别 UINTC 的设备树信息
    \item 接收方陷入内核前的状态的保存与恢复
    \item 注册并实现系统调用
\end{itemize}

\subsection{UINTC 驱动}

在 QEMU 的 virt 硬件环境中添加设备树生成代码,最后生成的设备树节点内容如下:

\lstset{basicstyle=\footnotesize\tt}
\begin{lstlisting}
uintc@2f10000 {
    interrupts-extended = <0x08 0x00 0x06 0x00 0x04 0x00 0x02 0x00>;
    reg = <0x00 0x2f10000 0x00 0x4000>;
    interrupt-controller;
    compatible = "riscv,uintc0";
};
\end{lstlisting}

其中各个字段的含义为:

\begin{itemize}
    \item \textbf{interrupts-extended} 连接到每个核 \Ruip 寄存器的 \FcsrUipUsip 位
    \item \textbf{interrupts-controller} 表示该设备是一个接收中断的控制器,这里加上是为了方便 Linux 识别,实际上 UINTC 并没有接收外部中断
    \item \textbf{compatible} 表示设备名称,Linux 内注册驱动时应该与之对应
\end{itemize}

驱动代码位于 drivers/irqchip/irq-riscv-uintc.c ,Linux 对 UINTC 进行的初始化过程如下:

\begin{itemize}
    \item 解析设备树获取外设对应的物理地址范围
    \item 初始化全局控制结构 \ref{code:uintcpriv} 并保存外设信息
    \item 调用 \mintinline[breaklines]{c}{void __iomem *ioremap(phys_addr_t addr, size_t size)} 完成内核物理地址到虚拟地址的映射,内核可以通过虚拟地址直接访问 UINTC 读写端口
    \item 初始化全局位图管理 UINTC 中已分配的槽位
\end{itemize}

\label{code:uintcpriv}
\begin{lstlisting}[style=CStyle]
    struct uintc_priv {
        struct cpumask lmask; // 位图记录已连接的 CPU
        void __iomem *regs;   // UINTC 在内核地址空间映射的起始地址
        resource_size_t size; // UINTC 地址范围大小
        u32 nr;               // UINTC 槽位数量
        void *mask;           // 位图记录已分配的槽位
        spinlock_t lock;      // 互斥锁
    };
\end{lstlisting}

\subsection{状态保存与恢复}

arch/riscv/kernel/entry.S 是 Linux 在 RISC-V 硬件平台上运行时的陷入入口,涉及上下文的保存与恢复、针对不同陷入原因跳转到对应的处理函数入口、针对不同系统调用号跳转到系统调用向量表对应的入口等。

Linux 中线程也被称为轻量级进程(LWP, Light-weight process)\cite{linuxbook},如无特殊说明,描述中默认使用进程指代进程或线程。考虑到在某些负载环境下,接收方进程可能在不同核上迁移,需要对用户态的控制状态进行保存与恢复,主要对 \Rutvec 、\Ruscratch、\Ruepc 三个寄存器进行保存。此外还需要在陷入时将接收方状态的 Active 位置 0 确保当这个核运行其他进程时不会被中断。接收方被唤醒并准备在某个核上运行前,除了对上述三个寄存器进行恢复外,还需要设置 \Rsuirs 寄存器,注意到当进程主动让权时,内核会两次进入 resume\_userspace 这个代码段,因此只能在 restore\_all 这个代码段重新将接收方状态的 Active 位置 1 确保被唤醒的是真正运行在这个核上的进程。

\subsection{进程管理}

在 arch/riscv/kernel/uintr.c 中添加用户态中断相关代码,主要涉及以下内容:

\begin{itemize}
    \item 控制结构定义和初始化:在进程控制块中维护发送方和接收方的状态
    \item 进程资源分配和回收:文件描述符,进程控制块内的状态,发送方状态表等
    \item 系统调用实现:在系统调用向量表中注册系统调用号,并通过 SYSCALL\_DEFINEx 宏声明并实现系统调用函数
\end{itemize}

\section{libc}

\subsection{库函数实现}

对系统调用进一步封装,包括设置 U 态 CSR 、上下文保存与恢复以及读取并更新 UINTC 中的 Pending Requests 等:

\begin{lstlisting}[style=CStyle]
    #include "uintr.h"

    extern void __handler_entry(struct __uintr_frame* frame, void* handler) {
        uint64_t irqs = uipi_read();
        csr_clear(CSR_UIP, MIE_USIE);
        uint64_t (*__handler)(struct __uintr_frame * frame, uint64_t) = handler;
        irqs = __handler(frame, irqs);
        uipi_write(irqs);
    }
    static uint64_t __register_receiver(void* handler) {
        // 设置中断处理函数入口
        csr_write(CSR_UTVEC, uintrvec);
        csr_write(CSR_USCRATCH, handler);
        // 使能 U 态中断处理
        csr_set(CSR_USTATUS, USTATUS_UIE);
        csr_set(CSR_UIE, MIE_USIE);
        int ret = __syscall0(__NR_uintr_register_receiver);
        // 使能 UINTC 
        uipi_activate();
        return ret;
    }
    // 用户调用接口
    #define uintr_register_receiver(handler) __register_receiver(handler)
\end{lstlisting}

注意到上述代码并没将用户注册的函数直接赋值给 \Rutvec 寄存器,而是赋值给了 \Ruscratch ,这是因为在汇编代码中需要先将通用寄存器保存在栈上,然后才能开始执行用户注册的函数。

\subsection{应用示例}

根据图 \ref{fig:uintr2} 中 RISC-V 用户态中断的工作流程,可以实现一个简单的进程间通信应用。

\begin{lstlisting}[style=CStyle]
    volatile unsigned int uintr_received; // volatile 避免编译优化
    unsigned int uintr_fd;

    uint64_t uintr_handler(struct __uintr_frame *ui_frame, uint64_t irqs) {
        uintr_received = 1; // 设置标志位
        return 0;
    }

    void *sender_thread(void *arg) {
        int uipi_index;
        // 注册发送方状态表项
        uipi_index = uintr_register_sender(uintr_fd);
        // 发送用户态中断
        uipi_send(uipi_index);
        return NULL;
    }

    int main() {
        pthread_t pt;
        int ret;
        // 注册接收方中断处理函数
        if (uintr_register_receiver(uintr_handler))
            exit(EXIT_FAILURE);
        // 注册文件描述符
        ret = uintr_create_fd(1);
        if (ret < 0) exit(EXIT_FAILURE);
        uintr_fd = ret;
        // 创建发送方
        if (pthread_create(&pt, NULL, &sender_thread, NULL))
            exit(EXIT_FAILURE);
        // 忙等待标志位
        while (!uintr_received);
        pthread_join(pt, NULL);
        close(uintr_fd);
        // 正常退出
        exit(EXIT_SUCCESS);
    }
\end{lstlisting}

接收方注册中断处理函数,注册文件描述符,创建发送方进程,并忙等待标志位;发送方进程注册后发送一个中断便直接退出;接收方收到中断并陷入中断处理函数后设置标志位,回到正常流程发现标志位已被设置,继续执行直到退出。