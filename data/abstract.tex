% !TeX root = ../thesis.tex

\begin{abstract}
  
  进程间通信是进程间协调合作的基础,其效率往往影响着系统整体的工作效率。尤其在微内核架构中,进程间通信是主要的性能瓶颈之一。传统的进程间通信机制如信号、管道等大多数需要内核的参与,不可避免地引入了切换的开销。共享内存机制虽然减少了切换的开销,但同时也引入了同步互斥的开销,且通信双方无法实时响应消息,仍需要轮询或其他机制辅助。
  
  用户态中断是近年来提出的进程间通信机制,由用户态直接收发中断信号,可以减少陷入内核的开销,而且独立的中断处理流程可以支持异步通信,确保实时性的同时减少等待的开销。本文基于 RISC-V 指令集架构提出了用户态中断扩展方案。不同于 x86 指令集架构的用户态中断扩展,本文的设计方案与 RISC-V 标准指令集结合紧密,并通过用户态中断控制器进一步减小了指令的访存开销。在硬件方面,修改 QEMU 模拟器验证设计方案并在 Rocket Chip 上实现该硬件扩展;在软件方面,修改 Linux 6.0 以适配硬件特性。最终在 FPGA 上搭建 SoC ,成功启动 Linux 并运行测试程序。实验表明,用户态中断相比于信号机制,可以取得 203 倍的性能提升;相比于 eventfd,可以取得 136 倍的性能提升。

  \thusetup{
    keywords = {用户态中断, 软硬件结合, 进程间通信},
  }
\end{abstract}

\begin{abstract*}

  IPC (Inter-Process Communication) is the basis of inter-process coordination and cooperation, and its efficiency often affects the efficiency of the overall work of the system. Especially in microkernel, IPC is one of the main performance bottlenecks. Most of the traditional IPC mechanisms such as Signal and Pipe require the participation of the kernel, which inevitably introduces context switching overhead. Although the shared memory mechanism reduces the overhead of context switching, it also introduces the overhead of synchronous mutual exclusion, and the communication ends cannot respond to messages in real time, so polling or other mechanisms are still needed.
  
  User-Interrupt is an IPC mechanism proposed in recent years. Programs in user mode can directly send and receive interrupt signals, which can reduce the overhead of trapping into the kernel, and the independent interrupt handler can support asynchronous communication, ensuring real-time performance and reducing waiting overhead. This paper proposes a User-Interrupt extension based on the RISC-V ISA. Unlike the User-Interrupt extension of the x86 ISA, the design in this article is closely integrated with the RISC-V standard ISA and further reduces the memory access overhead of instructions through a User-Interrupt Controller. In terms of hardware, the QEMU simulator is modified to verify the design and the hardware extension is implemented on Rocket Chip. In terms of software, Linux 6.0 is modified to adapt to hardware features. Finally we build SoC on FPGA with Rocket Chip and successfully boot Linux. Experiments show that compared with Signal, User-Interrupt can achieve 203x performance improvement; compared with eventfd, it can achieve 136x performance improvement.

  \thusetup{
    keywords* = {User-Interrupt, Hardware-Software Co-Design, IPC},
  }
\end{abstract*}
