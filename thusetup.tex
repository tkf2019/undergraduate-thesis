% !TeX root = ./thesis.tex

% 论文基本信息配置

\thusetup{
  output = electronic,
  % output = print,
  title  = {{基于 RISC-V 的用户态中断扩展}},
  title* = {RISC-V User-Interrupt},
  department = {计算机科学与技术系},
  discipline  = {计算机科学与技术},
  discipline* = {Computer Science and Technology},
  author  = {田凯夫},
  author* = {Tian Kaifu},
  supervisor  = {陈渝, 副教授},
  supervisor* = {Associate-Professor Chen Yu},
  %
  % 副指导教师
  %
  % associate-supervisor  = {陈文光, 教授},
  % associate-supervisor* = {Professor Chen Wenguang},
  %
  % 联合指导教师
  %
  % co-supervisor  = {某某某, 教授},
  % co-supervisor* = {Professor Mou Moumou},
  %
  % 日期
  %   使用 ISO 格式;默认为当前时间
  %
  % date = {2019-07-07},
  %
  % 是否在中文封面后的空白页生成书脊(默认 false)
  %
  include-spine = false,
}

% 载入所需的宏包

% 定理类环境宏包
\usepackage{amsthm}
% 也可以使用 ntheorem
% \usepackage[amsmath,thmmarks,hyperref]{ntheorem}

\thusetup{
  %
  % 数学字体
  % math-style = GB,  % GB | ISO | TeX
  math-font  = xits,  % sitx | xits | libertinus
}

% 可以使用 nomencl 生成符号和缩略语说明
% \usepackage{nomencl}
% \makenomenclature

% 表格加脚注
\usepackage{threeparttable}

% 表格中支持跨行
\usepackage{multirow}

% 固定宽度的表格。
% \usepackage{tabularx}

% 跨页表格
\usepackage{longtable}

% 算法
\usepackage{algorithm}
\usepackage{algorithmic}

% 量和单位
\usepackage{siunitx}

% 参考文献使用 BibTeX + natbib 宏包
% 顺序编码制
\usepackage[sort]{natbib}
\bibliographystyle{thuthesis-numeric}

% 著者-出版年制
% \usepackage{natbib}
% \bibliographystyle{thuthesis-author-year}

% 本科生参考文献的著录格式
% \usepackage[sort]{natbib}
% \bibliographystyle{thuthesis-bachelor}

% 参考文献使用 BibLaTeX 宏包
% \usepackage[style=thuthesis-numeric]{biblatex}
% \usepackage[style=thuthesis-author-year]{biblatex}
% \usepackage[style=apa]{biblatex}
% \usepackage[style=mla-new]{biblatex}
% 声明 BibLaTeX 的数据库
% \addbibresource{ref/refs.bib}

% 定义所有的图片文件在 figures 子目录下
\graphicspath{{figures/}}

% 数学命令
\makeatletter
\newcommand\dif{%  % 微分符号
  \mathop{}\!%
  \ifthu@math@style@TeX
    d%
  \else
    \mathrm{d}%
  \fi
}
\makeatother

% hyperref 宏包在最后调用
\usepackage{hyperref}

% 代码块配置
\usepackage{listings}
\usepackage{xcolor}

\definecolor{mGreen}{rgb}{0,0.6,0}

\lstdefinestyle{CStyle}{
    language=C,
    basicstyle=\footnotesize\tt,
    numbers=left,                
    numberstyle=\tiny\color{gray},
    numbersep=5pt,
    frame=lines,
    keywordstyle=\color{blue},
    commentstyle=\color{mGreen},
    stringstyle=\color{green},
    breakatwhitespace=false,
    breaklines=true,
    captionpos=b,
    keepspaces=true,
    showspaces=false,
    showstringspaces=false,
    showtabs=false,
}

% language definition
\lstdefinelanguage[RISC-V]{Assembler}
{
  alsoletter={.}, % allow dots in keywords
  alsodigit={0x}, % hex numbers are numbers too!
  morekeywords=[1]{ % instructions
    lb, lh, lw, lbu, lhu,
    sb, sh, sw,
    sll, slli, srl, srli, sra, srai,
    add, addi, sub, lui, auipc,
    xor, xori, or, ori, and, andi,
    slt, slti, sltu, sltiu,
    beq, bne, blt, bge, bltu, bgeu,
    j, jr, jal, jalr, ret,
    scall, break, nop
  },
  morekeywords=[2]{ % sections of our code and other directives
    .align, .ascii, .asciiz, .byte, .data, .double, .extern,
    .float, .globl, .half, .kdata, .ktext, .set, .space, .text, .word
  },
  morekeywords=[3]{ % registers
    zero, ra, sp, gp, tp, s0, fp,
    t0, t1, t2, t3, t4, t5, t6,
    s1, s2, s3, s4, s5, s6, s7, s8, s9, s10, s11,
    a0, a1, a2, a3, a4, a5, a6, a7,
    ft0, ft1, ft2, ft3, ft4, ft5, ft6, ft7,
    fs0, fs1, fs2, fs3, fs4, fs5, fs6, fs7, fs8, fs9, fs10, fs11,
    fa0, fa1, fa2, fa3, fa4, fa5, fa6, fa7
  },
  morecomment=[l]{;},   % mark ; as line comment start
  morecomment=[l]{\#},  % as well as # (even though it is unconventional)
  morestring=[b]",      % mark " as string start/end
  morestring=[b]'       % also mark ' as string start/end
}

\lstdefinestyle{ASMStyle}{
  language=[RISC-V]Assembler,
  basicstyle=\footnotesize\tt,
  numbers=left,                
  numberstyle=\tiny\color{gray},
  numbersep=5pt,
  frame=lines,
  keywordstyle=\color{blue},
  commentstyle=\color{mGreen},
  stringstyle=\color{green},
  breakatwhitespace=false,
  breaklines=true,
  captionpos=b,
  keepspaces=true,
  showspaces=false,
  showstringspaces=false,
  showtabs=false,
}

% 行内代码
\usepackage{minted}
\usemintedstyle{vs}

% 字符串宏定义
\usepackage{xspace}

\newcommand{\inst}[1]{{\tt #1}\xspace}
\newcommand{\reg}[1]{{\tt #1}\xspace}
\newcommand{\field}[1]{{\tt #1}\xspace}
\newcommand{\defregname}[2]{\providecommand{#1}{\reg{#2}}}
\newcommand{\deffieldname}[2]{\providecommand{#1}{\field{#2}}}
\newcommand{\definst}[2]{\providecommand{#1}{\inst{#2}}}

\deffieldname{\FcsrMstatusMprv}{MPRV}
\deffieldname{\FcsrMstatusMie}{MIE}
\deffieldname{\FcsrSstatusSie}{SIE}
\deffieldname{\FcsrUstatusUpie}{UPIE}
\deffieldname{\FcsrUstatusUie}{UIE}
\deffieldname{\FcsrUieUsie}{USIE}
\deffieldname{\FcsrUipUsip}{USIP}
\defregname{\Rmstatus}{mstatus}
\defregname{\Rsstatus}{sstatus}
\defregname{\Rmcause}{mcause}
\defregname{\Rmie}{mie}
\defregname{\Rmip}{mip}
\defregname{\Rxepc}{xepc}
\defregname{\Rmepc}{mepc}
\defregname{\Rsuirs}{suirs}
\defregname{\Rsuist}{suist}
\defregname{\Rustatus}{ustatus}
\defregname{\Ruie}{uie}
\defregname{\Ruip}{uip}
\defregname{\Rutvec}{utvec}
\defregname{\Ruepc}{uepc}
\defregname{\Ruscratch}{uscratch}
\defregname{\Rucause}{ucause}
\defregname{\Rutval}{utval}
\defregname{\Ruepc}{uepc}
\defregname{\Rsideleg}{sideleg}
\defregname{\Rsedeleg}{sedeleg}
\definst{\Iuret}{uret}
\definst{\Isret}{sret}
\definst{\Imret}{mret}
\definst{\Iuipi}{uipi}
\definst{\Iuipisend}{uipi.send}
\definst{\Iuipiread}{uipi.read}
\definst{\Iuipiwrite}{uipi.write}
\definst{\Iuipiact}{uipi.activate}
\definst{\Iuipideact}{uipi.deactivate}
\definst{\Ild}{ld}
\definst{\Isd}{sd}
